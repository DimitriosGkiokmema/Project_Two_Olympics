\documentclass[fontsize=11pt]{article}
\usepackage{amsmath}
\usepackage[utf8]{inputenc}
\usepackage[margin=0.75in]{geometry}

\title{CSC111 Project 2 Proposal: 

The Analysis of Summer Olympics Through External Effects 

(1940 - 2020)}
\author{Anh Dang Phuong, Dimitrios Gkiokmema, Dora Gombar, Maya Edri}
\date{\today}

\begin{document}
\maketitle

\section*{1., Problem Description and Research Question}

\textit{How did external - geopolitical, and societal - factors influence the outcomes and dynamics of the Olympic Games?}
\\
\\
Over the last 100 years, many historical occurrences have created tension while trying to perform in the Olympic Games, leaving an indelible mark on its history, fundamentally reshaping the competition landscape, and altering the trajectory of athletic achievement. 
\\
\\
Take, for instance, the impact of World Wars on the Olympics. Since the inception of the first modern Olympic Games in 1896, the sports game has faced cancellation on only three occasions: once during World War I (1916) and twice during World War II (1940, 1944). Yet, even in the post-war era, the scars of conflict lingered, leading to the decision to ban German and Japanese athletes from participating in 1948. 
\\ Afterward, during the Cold War, the Olympics became a battleground for ideological supremacy between the United States and the Soviet Union. The intense rivalry between the superpowers spilled onto the athletic stage, with each nation leveraging sporting success to bolster their respective political agendas. Following the Soviet invasion of Afghanistan, tensions between the United States and the Soviet Union escalated, leading to President Jimmy Carter's announcement of a boycott of the 1980 Moscow Summer Games by the United States. In response, the Soviet Union boycotted the 1984 Summer Olympics in Los Angeles.
\\ Finally, the fall of the Soviet Union in 1991 was a watershed moment in modern history, and its impact rippled across various facets of global affairs, including the Olympic competition. The Soviet Union formally dissolved and broke into fifteen separate nations, which altered the Olympic community's balance of power and presented logistical challenges as new nations tried to establish their sporting infrastructure.
\\
\\
The abovementioned examples - which are only fragments of the whole picture - perfectly illustrate the intricate background of the Olympic Games, indicating that sports results come not only from human capital but also from geopolitical and societal factors - often foreseen. Our project aims to emphasize, represent, and visualize such (international) historical events through the lens of our statistical computation approach: we plan to use graphical tree representations, pie and bar charts, and graphs.

\section*{2., Computational Plan}

TODO

 • Total medals in a given year

 • Total medals for a given country

 • Compare the number of Gold, Silver, and Bronze

 • Ranking (i.e. which country ranked the $i^{th}$ place for the number of (G/S/B/total) medals in year X?)

• Calculate the average duration of a country's success by measuring the number of consecutive editions in which a country wins medals. 

• Host Country Effect:
Analyze the impact of hosting the Olympics on a country's performance. Calculate the difference in medal counts for host countries in the year they hosted compared to non-host years.

• Team vs. Individual Sports Impact:
Analyze the impact of team sports versus individual sports on a country's overall medal count. Identify countries that excel in one category over the other.


\section*{3., References}

TODO

% NOTE: LaTeX does have a built-in way of generating references automatically,
% but it's a bit tricky to use so you are allowed to write your references manually
% using a standard academic format like MLA or IEEE.
% See project proposal handout for details.

\end{document}

